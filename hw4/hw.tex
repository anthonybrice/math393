\documentclass{abrice}

\title{Math 393: Homework 4}
\author{Anthony Brice}

\renewcommand\plaintitle{}

\usepackage[mathscr]{eucal}
\usepackage{xspace}

\newcommand{\GL}{\text{GL}}
\newcommand{\Z}{\mathbb{Z}}
\newcommand{\R}{\mathbb{R}}
\newcommand{\Q}{\mathbb{Q}}
\renewcommand{\C}{\mathbb{C}}
\renewcommand{\H}{\mathbb{H}}
\renewcommand{\P}{\mathscr{P}}

\newcommand{\Claim}{\emph{Claim.}\xspace}%

\usepackage{amsthm}

\usepackage{polynom}

\begin{document}
\maketitle

\section{Exercise 8.5}

Consider the subgroups $\langle 2 \rangle, \langle 3 \rangle$, and $\langle 2,3
\rangle$ of $\Q_{>0}$.

\subsection{Part a)}

\Claim $\langle 2 \rangle$ is isomorphic to $\langle 3 \rangle$.

\begin{proof}
  Let $r_n : \Z \to \langle n \rangle$ be defined by $r_n(x) = n^x$ where
  $\langle n \rangle$ is the subgroup of $\Q_{>0}$ generated by $n$. Then let
  $q_n : \langle n \rangle \to \Z$ be defined by $q_n(y) = \log_n y$. Then for
  $x \in \Z$ and $y \in \langle n \rangle$,
  \begin{align*}
    r_n(q_n(y))
    &= r_n(\log_n y) \\
    &= n^{\log_n y} \\
    &= y
  \end{align*}
  and
  \begin{align*}
    q_n(r_n(x))
    &= q_n(n^x) \\
    &= log_n (n^x) \\
    &= x\, .
  \end{align*}
  Thus $q_n = {r_n}^{-1}$, so $r_n$ is a bijection.

  Next we show that $r_n$ is operation preserving. For $a,b \in \Z$,
  \begin{align*}
    r_n(a + b)
    &= n^{a + b} \\
    &= n^a \cdot n^b \\
    &= r_n(a) \cdot r_n(b)\, .
  \end{align*}
  % Now for $a,b \in \langle n \rangle$,
  % \begin{align*}
  %   {r_n}^{-1}(a b)
  %   &= \log_n(ab) \\
  %   &= \log_n a + \log_n b \\
  %   &= {r_n}^{-1}(a) + {r_n}^{-1}(b)\, .
  % \end{align*}

  Thus $r_n$ is an isomorphism from $\Z$ to $\langle n \rangle$, so $r_2$ and
  $r_3$ are isomorphisms from $\Z$ to $\langle 2 \rangle$ and $\langle 3
  \rangle$ respectively. Then an isomorphism from $\langle 2 \rangle$ to
  $\langle 3 \rangle$ is simply $r_3 \circ {r_2}^{-1}$.
\end{proof}

\subsection{Part b)}

\Claim $\langle 2 \rangle$ is not isomorphic to $\langle 2,3 \rangle$.

\begin{proof}
  % Assume $f : \langle 2 \rangle \to \langle 2, 3 \rangle$ defines an
  % isomorphism. Then $f$ is a bijection and operation preserving.

  % Let $r_n : \Z \to \langle n \rangle$
  Note that $\langle 2 \rangle$ is a subgroup that is a cyclic group. Then since
  the property of being a cyclic group is preserved under isomorphisms, it
  suffices to show that $\langle 2, 3 \rangle$ is not cyclic.

  % Assume for the purpose of contradiction that $\langle 2,3 \rangle$ is cyclic.
  % Then there exists $a \in \langle 2,3 \rangle$ such that $\langle 2,3 \rangle =
  % \langle a \rangle$. Since $\langle a \rangle = \{ a^n : n \in \Z \}$, there
  % exist $k,j \in \Z$ such that $a^k = 2$ and $a^j = 3$. Then $a = \sqrt[k]2 =
  % \sqrt[j] 3$, and $a^k a^j = 2 \cdot 3 \Longleftrightarrow 6 = a^{k + j}$.
  % Therefore
  Let $f : \Z \times \Z \to \langle 2,3 \rangle$ be defined by $f(n,m) = 2^n
  3^m$. Then clearly $f$ is onto, and $f$ is one-to-one by the fundamental
  theorem of arithmetic. Thus $f$ is a bijection. Now note that for $(a,b),(c,d)
  \in \Z \times \Z$,
  \begin{align*}
    f((a,b) + (c,d))
    &= f(a+c, b + d) \\
    &= 2^{a+c} 3^{b+d} \\
    &= 2^a 2^c 3^b 3^d \\
    &= 2^a 3^b 2^c 3^d \\
    &= f(a,b) \cdot f(c,d)\, .
  \end{align*}
  Thus $f$ is operation preserving, so we have that $f$ is an isomorphism from
  $\Z \times \Z$ to $\langle 2,3 \rangle$.

  Since Exercise 8.10 proves that $\Z \times \Z$ is not cyclic, we must have
  that $\langle 2,3 \rangle$ is not cyclic. Then $\langle 2 \rangle$ cannot be
  isomorphic to $\langle 2,3 \rangle$.
\end{proof}

\section{Exercise 8.10}

\Claim The group $\Z \times \Z$ is not isomorphic to $\Z$.

\begin{proof}
  Since $\Z = \langle 1 \rangle$ and is thus cyclic, it suffices to show that
  $\Z \times \Z$ is not cyclic.

  Assume for the purpose of contradiction that $\Z \times \Z$ is cyclic. Then
  there exists $(n,m) \in \Z \times \Z$ such that $\Z \times \Z = \langle (n,m) \rangle$.
  Then we must have $x \in \Z$ such that $x \cdot (n,m) = (1,1) \Rightarrow xn =
  1 = xm \Rightarrow n = m \neq 0$. We must also have $y \in \Z$ such that $y
  \cdot (n,m) = (0,1) \Rightarrow yn = 0, ym = 1 \Rightarrow y = 0$ and $y \neq
  0$, so we have found a contradiction. Thus $\Z \times \Z$ is not cyclic and so
  cannot be isomorphic to $\Z$.
\end{proof}

\section{Exercise 8.11}

Let $G, G'$, and $G''$ be groups.

\subsection{Part 1)}

\Claim Let $f : G \to G'$ and $g : G' \to G''$ be isomorphisms of groups. Then
so is the composition $g \circ f : G \to G''$.

\begin{proof}
  First note that since $g$ and $f$ are bijections, the composition $f^{-1}
  \circ g^{-1}$ exists. Then for $x \in G$ and $y \in G''$,
  \begin{align*}
    (g \circ f)((f^{-1} \circ g^{-1})(y))
    &= (g \circ f)(f^{-1}(g^{-1}(y))) \\
    &= g(f(f^{-1}(g^{-1}(y)))) \\
    &= g(g^{-1}(y)) \\
    &= y
  \end{align*}
  and
  \begin{align*}
    (f^{-1} \circ g^{-1})((g \circ f)(x))
    &= (f^{-1} \circ g^{-1})(g(f(x))) \\
    &= f^{-1}(g^{-1}(g(f(x)))) \\
    &= f^{-1}(f(x)) \\
    &= x\, .
  \end{align*}
  Thus ${(g \circ f)}^{-1}$ is simply $f^{-1} \circ g^{-1}$, so $g \circ f$ is a
  bijection.

  We must now show that $g \circ f$ is operation preserving. For $a,b \in G$,
  \begin{align*}
    (g \circ f) (a *_G b)
    &= g(f(a *_G b)) \\
    &= g(f(a) *_{G'} f(b)) \\
    &= g(f(a)) *_{G''} g(f(b)) \\
    &= (g \circ f)(a) *_{G''} (g \circ f)(b)\, .
  \end{align*}

  Thus $g \circ f$ is a bijection and operation preserving, so $g \circ f$ is an
  isomorphism.
\end{proof}

\subsection{Part 2)}

\Claim Let $f : G \to G'$ be an isomorphism of groups. Then so is the inverse
function $f^{-1} : G' \to G$.

\begin{proof}
  Note that $f^{-1}$ is a bijection by definition. Then we only have left to
  show that $f^{-1}$ is operation preserving. For $a,b \in G'$,
  \begin{align*}
    f(f^{-1}(a) *_{G} f^{-1}(b))
    &= f(f^{-1}(a)) *_{G'} f(f^{-1}(b)) \\
    &= a *_{G'} b\, .
  \end{align*}
  Then $f(f^{-1}(a) *_{G} f^{-1}(b)) = a *_{G'} b \Rightarrow f^{-1}(a)
  *_{G} f^{-1}(b) = f^{-1}(a *_{G'} b)$.

  Since $f^{-1}$ is bijective and operation preserving, we have that $f^{-1}$ is
  an isomorphism of groups.
\end{proof}

\subsection{Part 3)}

\Claim The identity function from $G$ to $G$ is an isomorphism of groups.

\begin{proof}
  Let $\id : G \to G$ be defined by the identity function. Then clearly $\id$ is
  bijective with its inverse being itself. Then we only have left to show that $\id$ is
  operation preserving. For $a,b \in G$,
  \begin{align*}
    \id(ab)
    &= ab \\
    &= \id(a) \id(b)\, .
  \end{align*}
  Thus $\id$ is an isomorphism of groups.
\end{proof}

\section{Exercise 8.12}

\Claim Let $f : G \to G'$ be operation preserving where $(G,*)$ and $(G',
\circ)$ are groups. Then we have that $f(e_G) = e_{G'}$ and $f(a^{-1}) =
{f(a)}^{-1}$ for every $a \in G$.

\begin{proof}
  Note that $f(e_G) = f(e_G * e_G) = f(e_G) \circ f(e_G)$. Therefore $f(e_G)
  \circ e_{G'} = f(e_G) \circ f(e_G) \Rightarrow e_{G'} = f(e_G)$.

  Now let $a \in G$. Then
  \begin{align*}
    f(a) \circ f(a^{-1})
    &= f(a * a^{-1}) \\
    &= f(e_G) \\
    &= e_{G'}\, .
  \end{align*}
  Thus $f(a^{-1})$ is the right inverse of $f(a)$. Similarly
  \begin{align*}
    f(a^{-1}) \circ f(a)
    &= f(a^{-1} * a) \\
    &= f(e_G) \\
    &= e_{G'}\, ,
  \end{align*}
  so $f(a^{-1})$ is the left inverse of $f(a)$. Since it is both the left and
  right inverse of $f(a)$, $f(a^{-1})$ is the inverse of $f(a)$; that is,
  $f(a^{-1}) = {f(a)}^{-1}$.
\end{proof}

\section{Exercise 8.14}

\Claim Let $f : G \to G'$ be an operation preserving map from a group $G$ to a
group $G'$. If $f$ is one-to-one, then $G$ is isomorphic to a subgroup of
$G'$.

\begin{proof}
  Let $f : G \to G'$ be a one-to-one operation preserving map where $(G,*)$ and
  $(G', \circ)$ are groups. Let $H$ be the range of $f$, so $f : G \to H$ is an
  operation preserving bijection. Then we only have left to show that $H$ is a
  subgroup.

  Exercise 8.12 proves that the identity element and inverses are in the range
  of $f$.

  Let $a,b \in H$. Then $f^{-1}(a) * f^{-1}(b) = f^{-1}(a \circ b)$. Since
  $f^{-1}(a \circ b) \in G$, we have that $f(f^{-1}(a \circ b)) = a \circ b \in
  H$, so $H$ is closed under $\circ$.

  Thus $f$ defines an isomorphism from $G$ to $H \leq G'$
\end{proof}

\section{Exercise 8.23}

Let $a$ be an element of a group $G$.

\subsection{Part a)}

\Claim Let $c_a : G \to G$ be defined by $c_a(x) = a^{-1} x a$. Then $c_a$ is an
isomorphism from $G$ to itself.

\begin{proof}
  Let $f_a : G \to G$ be defined by $f_a(x) = a x a^{-1}$. Then for $x,y \in G$,
  \begin{align*}
    f_a(c_a(y))
    &= f_a(a^{-1} y a) \\
    &= a a^{-1} y a a^{-1} \\
    &= y
  \end{align*}
  and
  \begin{align*}
    c_a(f_a(x))
    &= c_a(a x a^{-1}) \\
    &= a^{-1} a x a^{-1} a \\
    &= x\, .
  \end{align*}
  Thus $f_a = {c_a}^{-1}$, so $c_a$ is a bijection.

  Next we show that $c_a$ is operation preserving. For $x,y \in G$,
  \begin{align*}
    c_a(xy)
    &= a^{-1} x y a \\
    &= a^{-1} x a a^{-1} y a \\
    &= c_a(x) c_a(y)\, .
  \end{align*}

  Thus $c_a$ is an isomorphism from $G$ to itself.
\end{proof}

\subsection{Part b)}

\Claim Let $c_a : G \to G$ be defined by $c_a(x) = a^{-1} x a$. Then $c_a \circ
c_b = c_{ab}$ for all $a,b \in G$.

\begin{proof}
  Let $b,x,y \in G$. Then
  \begin{align*}
    c_{ab}(xy)
    &= (ab)^{-1} xy ab \\
    &= b^{-1} a^{-1} xy ab \\
    &=
  \end{align*}

  \begin{align*}
    (c_a \circ c_b)(xy)
    &= c_a(c_b(xy)) \\
    &= c_a(b^{-1} xy b) \\
    &= a^{-1} b^{-1} xy b a
  \end{align*}
\end{proof}

\section{Exercise 8.24}

\subsection{Part a)}

\Claim Let $G$ be a group. Then the set $\Aut(G)$ of all automorphisms of $G$ is
a group under composition of functions.

\begin{proof}
  Composition is associative as shown below. For $h,g,f \in \Aut(G)$,
  \begin{align*}
    (h \circ (g \circ f))(x)
    &= h(g(f(x))) \\
    &= h \circ g(f(x)) \\
    &= (h \circ g) \circ f(x) \\
    &= ((h \circ g) \circ f) (x)\, .
  \end{align*}

  Let $\id{} : G \to G$ be the identity function. Exercise 8.11 proved $\id{} \in
  \Aut(G)$. Note that for any $f \in \Aut(G)$, $f \circ \id{} = f = \id{} \circ f$,
  so $\id{}$ is the identity element.

  Let $f \in \Aut(G)$. Since $f$ is an automorphism, its inverse $f^{-1}$ is in
  $\Aut(G)$ as well. Then $f \circ f^{-1} = \id{} = f^{-1} \circ f$.

  Since we have an associative binary operation, an identity element, and
  inverses, $\Aut(G)$ is a group under composition of functions.
\end{proof}

\subsection{Part b)}

\Claim The automorphisms of $\Z_3$ are the identity function and the function $f : \Z_3
\to \Z_3$ defined by $f(x) = x + x$.

\begin{proof}
  Clearly the identity function is bijective and operation preserving.

  $f$ is clearly bijective. Let $x \in \Z_3$. Then
  \begin{align*}
    f(x + x)
    &= f(2x) \\
    &= 4x \\
    &= 2x + 2x \\
    &= f(x) + f(x)\, .
  \end{align*}
  We exhaustively check the remaining cases:
  \begin{align*}
    &f(0 + 1) = f(1) = 2 = 0 + 2 = f(0) + f(1)\, , \\
    & f(1 + 2) = f(0) = 0 = 2 + 1 = f(1) + f(2)\, .
  \end{align*}
  Thus we have verified that $f$ is operation preserving.

  No other bijective mapping preserving the identity element exists.
\end{proof}

\noindent
\Claim The automorphisms of $Z_4$ are the identity function and the function $f
: \Z_4 \to \Z_4$ defined by
\[
  f(x) =
  \begin{cases}
    x &\text{if } x = 0 \text{ or } 2\\
    x + 2 &\text{if } x = 1 \text{ or } 3\, .
  \end{cases}
\]

\begin{proof}
  Clearly the identity function is bijective and operation preserving.

  $f$ is clearly bijective. To begin to show that $f$ is operation preserving, let $x \in
  \Z_4$ and note that $2x = 0$ or $2$. Then
  \begin{align*}
    f(x + x)
    &= f(2x) \\
    &= 2x \\
    &= f(x) + f(x)\, .
  \end{align*}
  We exhaustively check the remaining cases:
  \begin{align*}
    &f(0 + 1) = f(1) = 3 = 0 + 3 = f(0) + f(1)\, , \\
    &f(0 + 3) = f(3) = 1 = 0 + 1 = f(0) + f(3)\, , \\
    &f(2 + 1) = f(3) = 1 = 3 + 2 = f(2) + f(1)\, , \\
    &f(2 + 3) = f(1) = 3 = 3 + 1 = f(2) + f(3)\, .
  \end{align*}
  Thus $f$ is operation preserving.

  Let $g : \Z_4 \to \Z_4$ be the only other bijective mapping preserving the
  identity element. Then $g$ is defined
  \[
    g(x) =
    \begin{cases}
      x &\text{if } x = 0 \\
      x + 1 &\text{otherwise.}
    \end{cases}
  \]
  Note that $g(1 + 3) = g(0) = 0 \neq g(1) + g(3) = 2 + 3 = 1$. Thus $g$ is not
  operation preserving.
\end{proof}


\subsection{Part c)}

\Claim Any automorphism $f$ of $\Z_n$ is equal to multiplication by $f(1) \bmod
n$; that is, $f(x) = f(1) \cdot_n x$ for all $x \in \Z_n$.

\begin{proof}
  Note that for any $\Z_n$, $1$ is a generator. We first prove that any
  isomorphism of groups must map any generator in the domain to a generator in the
  codomain.

  Let $f$ be an isomorphism from group $G$ to group $G'$, and let $x$ be a
  generator in $G$. Then clearly $\langle f(x) \rangle \subset G'$. Now let $y
  \in G'$. Since $f$ is onto, there exists $a \in G$ such that $f(a) = y$. Since
  $G = \langle x \rangle$, $a = x^k$ for some $k \in \Z$. Then $f(a) = f(x^k) =
  {(f(x))}^k = y$, so $G' \subset \langle f(x) \rangle$. Since $\langle f(x)
  \rangle \subset G'$ and $G' \subset \langle f(x) \rangle$, $\langle f(x)
  \rangle = G'$, so $f(x)$ is a generator in $G'$. Thus any isomorphism of
  groups must map any generator in the domain to a generator in the codomain.

  Consider the base case $n = 1$. Then our group is the trivial group, and the
  only automorphism is the identity function. Then for all $x \in \Z_1$,
  \begin{align*}
    f(x)
    &= f(1) \cdot_1 x \\
    &= 0 \cdot_1 x \\
    &= 0\, .
  \end{align*}
  Thus the claim is true when $n = 1$.

  Assume the claim is true when $n = k$ for some $k \geq 1$. Then consider the
  case when $n = k + 1$. Then for all $x \in \Z_{k+1}$,
  \begin{align*}
    f(x) = f(1) \cdot_{k+1} x
  \end{align*}
\end{proof}

\end{document}

%  LocalWords:  isomorphisms isomorphism bijective bijection automorphism
%  LocalWords:  automorphisms codomain
