\documentclass{abrice}

\title{Math 393: Homework 3}
\author{Anthony Brice}

\usepackage[mathscr]{eucal}

\newcommand{\GL}{\text{GL}}
\newcommand{\Z}{\mathbb{Z}}
\newcommand{\R}{\mathbb{R}}
\newcommand{\Q}{\mathbb{Q}}
\renewcommand{\C}{\mathbb{C}}
\renewcommand{\H}{\mathbb{H}}
\renewcommand{\P}{\mathscr{P}}

\usepackage{amsthm}

\usepackage{polynom}

\begin{document}
\maketitle

\section{Exercise 18.2}

\emph{Claim.} Let $a$ and $b$ be elements of commutative ring $R$. Then ${(a + b)}^2 = a^2
+ 2 \cdot (ab) + b^2$.

\begin{proof}
  \begin{align*}
    {(a + b)}^2
    &= (a + b) \cdot (a + b) \\
    &= a^2 + ab + ba + b^2 \\
    &= a^2 + 2 \cdot (ab) + b^2\, . \qedhere
  \end{align*}
\end{proof}

\noindent
\emph{Claim.} Let $a$ and $b$ be elements of non-commutative ring $R$. Then it
does not necessarily hold that ${(a + b)}^2 = a^2 + 2 \cdot (ab) + b^2$.

\begin{proof}
  Let $a = \left(\begin{smallmatrix} 1&1 \\ 0&1 \end{smallmatrix} \right)$ and
  $b = \left( \begin{smallmatrix} 0&1 \\ 0&1 \end{smallmatrix} \right)$, both in
  $M_2(\Q)$. Then
  \begin{align*}
    {(a + b)}^2
    &= (a + b) \cdot (a + b) \\
    &= a^2 + ab + ba + b^2 \\
    &= \left(  \begin{smallmatrix} 1&2 \\ 0&1 \end{smallmatrix} \right)
      + \left( \begin{smallmatrix} 0&2 \\ 0&1 \end{smallmatrix} \right)
      + \left( \begin{smallmatrix} 0&1 \\ 0&1 \end{smallmatrix} \right)                                       + \left( \begin{smallmatrix} 0&1 \\ 0&1 \end{smallmatrix} \right) \\
    &\neq a^2 + 2 \cdot (ab) + b^2\, . \qedhere
  \end{align*}
\end{proof}

\section{Exercise 18.6}

\emph{Claim.} Every subring of an integral domain is an integral domain.

\begin{proof}
  Let $S$ be a subring of ring $R$ where $R$ is an integral domain. Then $R$ is
  a ring that is a subring of some field $T$. Then $S$ is also a subring of $T$,
  thus satisfying the definition of an integral domain.
\end{proof}

\section{Exercise 18.13}

\emph{Claim.} For any $x$ and $y$ in integral domain $R$, $x^2 = y^2$ if and
only if $x = y$ or $x = -y$.

\begin{proof}
  Let $x,y \in R$ where $R$ is an integral domain.

  Assume $x = y$. Then $x^2 = xx = yy = y^2$. Similarly for $x = -y$, $x^2 = xx
  = (-y)(-y) = y^2$ by Proposition 18.18.

  Assume $x^2 = y^2$. Then $x^2 - y^2 = 0$. Since this is a difference of
  squares, we have that $(x + y)(x - y) = 0 \Rightarrow x = y$ or $x = -y$.
\end{proof}

\section{Exercise 18.15}

Let $R = \{a/b : a,b \in \Z \text{ and } b \text{ is odd}\}$.

\subsection{Part a)}

\emph{Claim.} $R$ is a subring of $\Q$.

\begin{proof}
  We want to show that the three conditions of Proposition 18.13 hold.

  Clearly $R$ is a subset of $\Q$. Let $x,y \in R$ and consider $z = x + y$.
  Since the least common multiple of two odd numbers is itself an odd number, $z
  \in R$ as well, so $R$ is closed under the group operation of $\Q$. Since $0
  \in \Z$ and $0/b$ where $b$ is odd equals $0$, we have the additive identity
  element in $R$. By definition, $R$ clearly gives us additive inverses. Thus
  $R$ is a subgroup of $\Q$.

  Since $1 \in \Z$ and is odd, we have the multiplicative identity in $R$.

  Now consider the product $w = xy$. Since the product of two odd numbers is odd
  and the product of any two integers is an integer, $w \in R$ and thus $R$ is
  closed under multiplication in $\Q$.
\end{proof}

\subsection{Part b)}

\emph{Claim.} $R^* = \{ a/b : a,b \in \Z \text{ and } a,b \text{ are odd} \}$.

\begin{proof}
  Let $a/b \in R$. Then the multiplicative inverse of $a/b$ is $b/a$. Then $b/a$
  is only in $R$ when $a$ is odd.
\end{proof}

\section{Exercise 18.18}

Let $\Z[\sqrt 2] = \{a + b\sqrt 2 : a,b \in \Z \}$.

\subsection{Part a)}

\emph{Claim.} $\Z[\sqrt 2]$ is a subring of $\R$.

\begin{proof}
  Let $x = a + b\sqrt 2$ and $y = c + d\sqrt{2}$ where $a,b,c,d \in \Z$. Then
  \begin{align*}
    x + y
    &= a + b\sqrt 2 + c + d\sqrt 2 \\
    &= a + c + (b + d)\sqrt 2\, .
  \end{align*}
  Thus $x + y \in \Z[\sqrt 2]$. Clearly $0 \in \Z[\sqrt 2]$, and we have
  additive inverses since for any $a,b \in \Z$ we have $-a, -b \in \Z$.

  Consider the element $1 + 0 \sqrt 2 = 1$. Then $1 \in \Z[\sqrt 2]$.

  Next consider the product
  \begin{align*}
    xy
    &= (a + b\sqrt 2) (c + d \sqrt 2) \\
    &= a c + a d \sqrt 2 + b c \sqrt 2 + (b \sqrt 2)(d \sqrt 2) \\
    &= a c + \sqrt 2 (ad + bc) + 2bd \\
    & = ac + 2bd + \sqrt 2(ad + bc)\, .
  \end{align*}
  Thus $xy \in \Z[\sqrt 2]$.
\end{proof}

\subsection{Part b)}

\emph{Claim.} $\Z[\sqrt 2]$ is the smallest subring of $\R$ containing $\sqrt
2$.

\begin{proof}
  Let $R$ be a subring of $\R$ containing $\sqrt 2$. Then $0,1,\sqrt 2 \in R$.
  Then since $R$ is an abelian subgroup of $\R$ under addition, $R$ contains at
  least all elements of the form $a + b\sqrt2 : a,b \in \Z$. Then any subring of
  $\R$ containing $\sqrt 2$ contains $\Z[\sqrt 2]$, and since we have already
  proved that $\Z[\sqrt 2]$ is a subring of $R$, we have that $\Z[\sqrt 2]$ is
  the smallest subring of $\R$ containing $\sqrt 2$.
\end{proof}

\subsection{Part c)}

\emph{Claim.} $1 + \sqrt 2$ is a unit in $\Z[\sqrt 2]$.

\begin{proof}
  \begin{align*}
    (1 + \sqrt 2)(\sqrt 2 - 1)
    &= \sqrt 2 - 1 + 2 - \sqrt 2 \\
    &= 1\, .
  \end{align*}
  Since $\sqrt 2 - 1 \in \Z[\sqrt 2]$, the element is the inverse of $1 + \sqrt 2$.
\end{proof}

\section{Exercise 18.20}

Let $\Z[\sqrt[3]{2}] = \{ a + b \sqrt[3]{2} + c \sqrt[3]{4} : a,b,c \in \Z \}$.

\subsection{Part a)}

\emph{Claim.} $\Z[\sqrt[3]{2}]$ is a subring of $\R$.

\begin{proof}
  Let $x = a + b \sqrt[3]{2} + c \sqrt[3]{4}$ and $y = e + f \sqrt[3]{2} + g
  \sqrt[3]{4}$ where $a,b,c,d,e,f \in \Z$. Then
  \begin{align*}
    x + y
    &= a + b \sqrt[3]{2} + c \sqrt[3]{4} + e + f \sqrt[3]{2} + g \sqrt[3]{4} \\
    &= a + e + \sqrt[3]{2}(b + f) + \sqrt[3]{4}(c + g)\, .
  \end{align*}
  So $x + y \in \Z[\sqrt[3]{2}]$. Clearly $0 \in \Z[\sqrt[3]{2}]$, and we have
  additive inverses since for any $a,b,c \in \Z$ we have $-a,-b,-c \in \Z$.

  Consider $1 + 0\sqrt[3]{2} + 0\sqrt[3]{4} = 1$. Then $1 \in \Z[\sqrt[3]{2}]$.

  Now consider the product
  \begin{align*}
    xy
    &= (a + b \sqrt[3]{2} + c \sqrt[3]{4})(e + f \sqrt[3]{2} + g \sqrt[3]{4}) \\
    &= ae + af \sqrt[3]{2} + ag \sqrt[3]{4} + be \sqrt[3]{2} + bf \sqrt[3]{4}
      + 2bg + ce \sqrt[3]{4} + 2cf + 2cg \sqrt[3]{2} \\
    &= ae + 2(bg + cf) + \sqrt[3]{2} (af + be + 2cg) + \sqrt[3]{4} (ag + bf + ce)\, .
  \end{align*}
  So $xy \in \Z[\sqrt[3]{2}]$.
\end{proof}

\subsection{Part b)}

\emph{Claim.} Let $S = \{ a + b \sqrt[3]{2} : a,b \in \Z \}$. Then $\sqrt[3]{4}
\notin S$.

\begin{proof}
  Assume $\sqrt[3]{4} \in S$. Then there exists $a,b \in \Z$ where
  \begin{align*}
    &a + b \sqrt[3]{2} = \sqrt[3]{4} \\
    \Longleftrightarrow \quad
    & \sqrt[3]{2}(a + b \sqrt[3]{2}) = \sqrt[3]{4} \cdot \sqrt[3]{2} \\
    \Longleftrightarrow \quad
    & a \sqrt[3]{2} + b \sqrt[3]{4} = 2 \\
    \Longleftrightarrow \quad
    & a \sqrt[3]{2} + b(a + b \sqrt[3]{2}) = 2 \\
    \Longleftrightarrow \quad
    & a \sqrt[3]{2} + ab + b^2 \sqrt[3]{2} = 2 \\
    \Longleftrightarrow \quad
    & (a + b^2)\sqrt[3]{2} = 2 - ab\, .
  \end{align*}
  Then on the right-hand side we have an integer, but on the left-hand side we
  can only have an integer when $a + b^2 = 0$. This leads to the following
  system of equations:
  \begin{IEEEeqnarray*}{c}
    a + b^2 = 0\, , \\
    2 - ab = 0\, .
  \end{IEEEeqnarray*}
  Note that the only $(a, b)$ solving the second equation are $(1,2)$, $(2,1)$,
  $(-1,-2)$, and $(-2,-1)$. Plugging each pair into the first equation shows we
  have found a contradiction.
  \begin{align*}
    1 + 2^2 &\neq 0\, , \\
    2 + 1^2 &\neq 0\, , \\
    -1 + {(-2)}^2 &\neq 0\, , \\
    -2 + {(-1)}^2 &\neq 0\, .
  \end{align*}
  Thus $\sqrt[3]{4} \notin S$.
\end{proof}

\subsection{Part c)}

\emph{Claim.} Let $S = \{a + b\sqrt[3]{2} : a,b \in \Z \}$. Then $S$ is not a
subring of $\R$.

\begin{proof}
  Consider that $\sqrt[3]{2} \cdot \sqrt[3]{2} = \sqrt[3]{4}$ is not in $S$.
  Then $S$ is not closed under multiplication in $\R$.
\end{proof}

\subsection{Part d)}

\emph{Claim.} $\Z[\sqrt[3]{2}]$ is the smallest subring of $\R$ containing
$\sqrt[3]{2}$.

\begin{proof}
  Let $R$ be a subring of $\R$ containing $\sqrt[3]{2}$. Then $0,1,\sqrt[3]{2}
  \in R$. Since $R$ is an abelian subgroup under addition in $\R$, and is closed
  under multiplication in $\R$, then $R$ contains at least all elements of the
  form $a + b\sqrt[3]{2} + c\sqrt[3]{4} : a,b,c \in \Z$. Then any subring of
  $\R$ containing $\sqrt[3]{2}$ contains $\Z[\sqrt[3]{2}]$, and since we have
  already proved that $\Z[\sqrt[3]{2}]$ is a subring of $\R$, we have that
  $\Z[\sqrt[3]{2}]$ must be the smallest subring of $\R$ containing
  $\sqrt[3]{2}$.
\end{proof}

\section{Exercise 18.28}

Let $a$ and $b$ be elements of a ring $R$ such that $ab = ba$.

\subsection{Part a)}

\emph{Claim.} ${(a + b)}^2 = a^2 + 2 \cdot (ab) + b^2$.

\begin{proof}
  \begin{align*}
    {(a + b)}^2
    &= (a + b) \cdot (a + b) \\
    &= a^2 + ab + ba + b^2 \\
    &= a^2 + 2 \cdot (ab) + b^2\, . \qedhere
  \end{align*}
\end{proof}

\subsection{Part b)}

\emph{Claim.} ${(a + b)}^3 = a^3 + 3 \cdot (a^2b) + 3 \cdot (ab^2) + b^3$.

\begin{proof}
  \begin{align*}
    {(a + b)}^3
    &= (a + b){(a + b)}^2 \\
    &= (a + b)(a^2 + 2ab + b^2) \\
    &= (a + b)(a^2 + ab + ab + b^2) \\
    &= a^3 + a^2 b + a^2 b + a b^2 + a^2 b + a b^2 + a b^2 + b^3 \\
    &= a^3 + 3 \cdot (a^2b) + 3 \cdot (ab^2) + b^3\, . \qedhere
  \end{align*}
\end{proof}

\subsection{Part c)}

\emph{Claim.} The binomial theorem
\[
  {(a + b)}^n = \sum_{k=0}^n {n \choose k} \cdot (a^k b^{n-k})
\]
holds for any positive integer $n$.

\begin{proof}
  Consider the base case $n = 1$. We have on the left-hand side $a + b$, and
  computing the right gives us
  \begin{align*}
    \sum_{k=0}^1 {1 \choose k} \cdot (a^k b^{1-k})
    &= {1 \choose 0} \cdot (a^0b^1) + {1 \choose 1} \cdot (a^1b^0) \\
    &= b + a \\
    &= a + b\, .
  \end{align*}

  Now assume for the purpose of induction that the claim holds when $n = m$ and
  consider the case when $n = m + 1$. Starting from the right-hand side we have
  \begin{align*}
    \sum_{k=0}^{m+1} {m + 1 \choose k} \cdot (a^k b^{m+1-k})
    &= \sum_{k=0}^{m+1} \left(  {m \choose k} + {m \choose k-1 } \right) \cdot
      (a^k b^{m+1-k}) \\
    &= \sum_{k=0}^{m+1} \left( {m \choose k} \cdot (a^k b^{m+1-k}) + {m \choose k-1}
      \cdot (a^k b^{m+1-k})  \right) \\
    &= \sum_{k=0}^{m+1} {m \choose k} \cdot (a^k b^{m+1-k}) + \sum_{j=0}^{m+1} {m \choose j-1}
      \cdot (a^j b^{m+1-j})\, .
  \end{align*}
  Since ${m \choose m + 1} = 0$ and ${m \choose -1} = 0$,
  \begin{align*}
    \sum_{k=0}^{m+1} {m + 1 \choose k} \cdot (a^k b^{m+1-k})
    &= \sum_{k=0}^{m} {m \choose k} \cdot (a^k b^{m+1-k}) + \sum_{j=1}^{m+1} {m \choose j-1}
      \cdot (a^j b^{m+1-j}) \\
    &= b \cdot \sum_{k=0}^{m} {m \choose k} \cdot (a^k b^{m-k}) +
      \sum_{j=1}^{m+1} {m \choose j-1}
      \cdot (a^j b^{m+1-j}) \\
    &= b \cdot {(a + b)}^m + \sum_{j=1}^{m+1} {m \choose j-1} \cdot (a^j
      b^{m+1-j}) \\
    &= b \cdot {(a + b)}^m + a \cdot \sum_{j=1}^{m+1} {m \choose j-1} \cdot
      (a^{j-1} b^{m-(j-1)}) \\
    &= b \cdot {(a + b)}^m + a \cdot \sum_{j=0}^{m} {m \choose j} \cdot
      (a^{j} b^{m-j}) \\
    &= b \cdot {(a + b)}^m + a \cdot {(a + b)}^m \\
    &= (b + a){(a + b)}^m \\
    &= {(a + b)}^{m+1}\, . \qedhere
  \end{align*}
\end{proof}

\section{Exercise 18.29}

Let $S$ be a set, and let $\P(S)$ be the set of all subsets of $S$. Define $A
\oplus B = A \cup B - A \cap B$ for any two sets $A$ and $B$.

\subsection{Part a)}

\emph{Claim.} $\P(S)$ is a ring where addition is $\oplus$ and multiplication
is $\cap$.

\begin{proof}
  We must first show that $\P(S)$ is an abelian group under $\oplus$. Let $A, B,
  C \in \P(S)$ and let $x \in (A \oplus B) \oplus C$. Then we have $2$ exclusive
  cases, $x \in A \oplus B$ or $x \in C$. For the case when $x \in A \oplus B$
  we have $2$ more exclusive cases, $x \in A$ or $x \in B$. Thus we have $3$
  exclusive cases and for each we will show that $x \in A \oplus (B \oplus C)$.
  \begin{enumerate}[label=\emph{\roman*})]
  \item ($x \in A$.) $x \notin B$ and $x \notin C \Rightarrow x \notin B \oplus
    C$. $x \notin B \oplus C$ and $x \in A \Rightarrow x \in A \oplus (B \oplus
    C)$.
  \item ($x \in B$.) $x \in B$ and $x \notin C \Rightarrow x \in B \oplus C$. $x
    \in B \oplus C$ and $x \notin A \Rightarrow x \in A \oplus (B \oplus C)$.
  \item ($x \in C$.) $x \in C$ and $x \notin B \Rightarrow x \in B \oplus C$.
    $x \in B \oplus C$ and $x \notin A \Rightarrow x \in A \oplus (B \oplus C)$.
  \end{enumerate}
  Then $(A \oplus B) \oplus C \subset A \oplus (B \oplus C)$.

  Now let $y \in A \oplus (B \oplus C)$. As above, we then have $3$ exclusive
  cases: $y \in A$, $y \in B$, or $y \in C$. For each we will show that $y \in
  (A \oplus B) \oplus C$.
  \begin{enumerate}[label=\emph{\roman*})]
  \item ($y \in A$.) $y \in A$ and $y \notin B \Rightarrow y \in A \oplus B$. $y
    \in A \oplus B$ and $y \notin C \Rightarrow y \in (A \oplus B) \oplus C$.
  \item ($y \in B$.) $y \in B$ and $y \notin A \Rightarrow y \in A \oplus B$. $y
    \in A \oplus B$ and $y \notin C \Rightarrow y \in (A \oplus B) \oplus C$.
  \item ($y \in C$.) $y \notin A$ and $y \notin B \Rightarrow y \notin A \oplus
    B$. $y \notin A \oplus B$ and $y \in C \Rightarrow y \in (A \oplus B) \oplus
    C$.
  \end{enumerate}
  Then $A \oplus (B \oplus C) \subset (A \oplus B) \oplus C$. Since we also have
  $(A \oplus B) \oplus C \subset A \oplus (B \oplus C)$, we must have that $(A
  \oplus B) \oplus C = A \oplus (B \oplus C)$, so $\oplus$ is associative.

  Now consider the empty set $\O$. $A \oplus \O = A \cup \O - A \cap \O = A = \O
  \cup A - \O \cap A = \O \oplus A$, so $\O$ is the additive identity. Next,
  note that $A \oplus A = A \cup A - A \cap A = \O$, so the additive inverse of
  any element in $\P(S)$ is itself. Finally since $\cup$ and $\cap$ are
  commutative, $A \oplus B = A \cup B - A \cap B = (A \cup B) \cap (A \cap B)' =
  (B \cup A) \cap (B \cap A)' = B \cup A - B \cap A = B \oplus A$ where $D'$
  defines the complement of $D$ for all $D \in \P(S)$. Thus $\oplus$
  is commutative.

  Since we have an associative and commutative operation, an identity element,
  and inverses, we have that $\P(S)$ is an abelian group under $\oplus$.

  Next we consider the multiplication operation $\cap$. Since $A \cap S = A = S
  \cap A$, $S$ is the multiplicative identity. $\cap$ is clearly associative.

  Now we only have left to show that $\cap$ is left- and right-distributive over
  $\oplus$. Let $x \in A \cap (B \oplus C)$. Then we have two exclusive cases:
  $x \in A \cap B$, or $x \in A \cap C$.
  \begin{enumerate}[label=\emph{\roman*})]
  \item ($x \in A \cap B$.) $x \notin A \cap C$ and $x \in A \cap B \Rightarrow
    x \in (A \cap B) \oplus (A \cap C)$.
  \item ($x \in A \cap C$.) $x \notin A \cap B$ and $x \in A \cap C \Rightarrow
    x \in (A \cap B) \oplus (A \cap C)$.
  \end{enumerate}
  Then $A \cap (B \oplus C) \subset (A \cap B) \oplus (A \cap C)$.

  Similarly let $y \in (A \cap B) \oplus (A \cap C)$. Then we again have two
  exclusive cases: $y \in A \cap B$, or $y \in A \cap C$.
  \begin{enumerate}[label=\emph{\roman*})]
  \item ($y \in A \cap B$.) $y \in A \cap B$ and $y \notin A \cap C \Rightarrow
    y \notin C$. $y \in A \cap B \Rightarrow y \in B$. $y \in B$ and $y \notin C
    \Rightarrow y \in B \oplus C$. $y \in A \cap B \Rightarrow y \in A$. $y \in
    A$ and $y \in B \oplus C \Rightarrow y \in A \cap (B \oplus C)$.
  \item ($y \in A \cap C$.) $y \in A \cap C$ and $y \notin A \cap B \Rightarrow
    y \notin B$. $y \in A \cap C \Rightarrow y \in C$. $y \notin B$ and $y \in C
    \Rightarrow y \in B \oplus C$. $y \in A \cap C \Rightarrow y \in A$. $y \in
    A$ and $y \in B \oplus C \Rightarrow y \in A \cap (B \oplus C)$.
  \end{enumerate}
  Then $(A \cap B) \oplus (A \cap C) \subset A \cap (B \oplus C)$. Since we also
  have $A \cap (B \oplus C) \subset (A \cap B) \oplus (A \cap C)$, we must have
  that $A \cap (B \oplus C) = (A \cap B) \oplus (A \cap C)$. Thus $\cap$ is
  left-distributive over $\oplus$.

  Now let $x \in (B \oplus C) \cap A$. Then we have two exclusive cases: $x \in
  B \cap A$ or $x \in C \cap A$.
  \begin{enumerate}[label=\emph{\roman*})]
  \item ($x \in B \cap A$.) $x \notin C \cap A$ and $x \in B \cap A \Rightarrow
    x \in (B \cap A) \oplus (C \cap A)$.
  \item ($x \in C \cap A$.) $x \notin B \cap A$ and $x \in C \cap A \Rightarrow
    x \in (B \cap A) \oplus (C \cap A)$.
  \end{enumerate}
  Then $(B \oplus C) \cap A \subset (B \cap A) \oplus (C \cap A)$.

  Similarly let $y \in (B \cap A) \oplus (C \cap A)$. Then we again have two
  exclusive cases: $y \in B \cap A$, or $y \in C \cap A$.
  \begin{enumerate}[label=\emph{\roman*})]
  \item ($y \in B \cap A$.) $y \in B \cap A$ and $y \notin C \cap A \Rightarrow
    y \notin C$. $y \in B \cap A \Rightarrow y \in B$. $y \in B$ and $y \notin C
    \Rightarrow y \in B \oplus C$. $y \in B \cap A \Rightarrow y \in A$. $y \in
    B \oplus C$ and $y \in A \Rightarrow y \in (B \oplus C) \cap A$.
  \item ($y \in C \cap A$.) $y \in C \cap A$ and $y \notin B \cap A \Rightarrow
    y \notin B$. $y \in C \cap A \Rightarrow y \in C$. $y \notin B$ and $y \in C
    \Rightarrow y \in B \oplus C$. $y \in C \cap A \Rightarrow y \in A$. $y \in
    B \oplus C$ and $y \in A \Rightarrow y \in (B \oplus C) \cap A$.
  \end{enumerate}
  Then $(B \cap A) \oplus (C \cap A) \subset (B \oplus C) \cap A$. Since we also
  have $(B \oplus C) \cap A \subset (B \cap A) \oplus (C \cap A)$, we must have
  that $(B \oplus C) \cap A = (B \cap A) \oplus (C \cap A)$. Thus $\cap$ is
  right-distributive over $\oplus$.

  Since all properties of Definition 18.1 are satisfied, we now conclude the proof.
\end{proof}

\subsection{Part b)}

Cayley tables for both operations are given in Tables~\ref{tab:29b1} and \ref{tab:29b2}.

\begin{table}
  \centering
  \begin{tabular}{r|llll}
    $\oplus$ & $\O$ & $\{a\}$ & $\{b\}$ & $\{a,b\}$ \\
    \midrule
    $\O$ & $\O$ & $\{a\}$ & $\{b\}$ & $\{a,b\}$ \\
    $\{a\}$ & $\{a\}$ & $\O$ & $\{a,b\}$ & $\{b\}$ \\
    $\{b\}$ & $\{b\}$ & $\{a,b\}$ & $\O$ & $\{a\}$ \\
    $\{a,b\}$ & $\{a,b\}$ & $\{b\}$ & $\{a\}$ & $\O$
  \end{tabular}
  \caption{The Cayley table for $\oplus$ in $\P(S)$ when $S = \{a,b\}$.}
  \label{tab:29b1}
\end{table}

\begin{table}
  \centering
  \begin{tabular}{r|llll}
    $\cap$ & $\O$ & $\{a\}$ & $\{b\}$ & $\{a,b\}$ \\
    \midrule
    $\O$ & $\O$ & $\O$ & $\O$ & $\O$ \\
    $\{a\}$ & $\O$ & $\{a\}$ & $\O$ & $\{a\}$ \\
    $\{b\}$ & $\O$ & $\O$ & $\{b\}$ & $\{b\}$ \\
    $\{a,b\}$ & $\O$ & $\{a\}$ & $\{b\}$ & $\{a,b\}$
  \end{tabular}
  \caption{The Cayley table for $\cap$ in $\P(S)$ when $S = \{a,b\}$.}
  \label{tab:29b2}
\end{table}

\section{Email Exercise}

\emph{Claim.} $(a + bi + cj + dk)(a - bi - cj -dk) = a^2 + b^2 + c^2 + d^2$ for
all $a,b,c,d \in \R$ as elements of the ring $\H$ of quaternions.

\begin{proof}
  \begin{IEEEeqnarray*}{+rCl+x*}
    (a + bi + cj + dk)(a - bi - cj - dk)
    &=& a^2 - abi - acj - adk \\
    && +\> b i a - b i b i - b i c j - b i d k \\
    && +\> c j a - c j b i - c j c j  - c j d k\\
    && +\> d k a - d k b i - d k c j - d k d k \\
    &=& a^2 + b^2 + c^2 + d^2 \\
    && -\> abi - acj - adk \\
    && +\> bia - bck + bdj \\
    && +\> cja + cbk - cdi \\
    && +\> dka - dbj + dci \\
    &=& a^2 + b^2 + c^2 +d^2\, . & \qedhere
  \end{IEEEeqnarray*}
\end{proof}

\section{Exercise 17.20}

\emph{Claim.} $\Q$ is the smallest subfield of $\R$.

\begin{proof}
  Let $F$ be a subfield of $\R$. Then $1 \in F$, and since $F$ is an abelian
  subgroup under addition, $F$ must contain $\Z$. Since $F - \{0\}$ is an
  abelian subgroup under multiplication in $\Z$, $F$ must contain the
  multiplicative inverses for all elements of $\Z - \{0\}$, so $F$ contains all
  elements of the form $1/a : a \in \Z - \{0\}$. Then since $F$ is again an
  abelian subgroup under addition, including elements of the form $1/a : a \in
  \Z - \{0\}$ implies that $F$ contains $\Q$. Then we only have left to show
  that $\Q$ is a subfield of $\R$.

  Let $a,b \in \Q$. Clearly $a + b \in \Q$, $0 \in \Q$, and for any $a \in \Q$, $-a
  \in \Q$. Then $\Q$ is a subgroup of $\R$ under addition in $\R$.

  Similarly, $ab \in \Q$, $1 \in \Q$, and for any $a \in \Q - \{0\}$, $1/a \in
  \Q - \{0\}$. Then $\Q - \{0\}$ is a subgroup of $\R - \{0\}$ under
  multiplication in $\R$.

  Thus $\Q$ is a subfield of $\R$, and since any subfield of $\R$ must contain
  $\Q$, it follows that $\Q$ is the smallest subfield of $\R$.
\end{proof}

\section{Exercise 17.21}

Let $\Q(i) = \{a + bi : a,b \in \Q \}$.

\subsection{Part a)}

\emph{Claim.} $\Q(i)$ is a proper subfield of $\C$.

\begin{proof}
  Clearly $\Q(i)$ is a strict subset of $\C$ since it contains the rationals in
  $\C$ but lacks the irrationals. Then it suffices to prove that $\Q(i)$ is a
  subfield of $\C$.

  We first show that $\Q(i)$ is a subgroup of $\C$ under addition in $\C$. Let
  $x = a + bi$ and $y = c + di$, where $a,b,c,d \in \Q$. Then
  \begin{align*}
    x + y
    &= a + bi + c + di \\
    &= a + c + i(b + d)\, .
  \end{align*}
  Since $a + c \in \Q$ and $b + d \in \Q$, $x + y \in \Q(i)$. Thus $\Q(i)$ is
  closed under addition in $\C$. Now note that the additive identity $0$ in $\C$
  is also in $\Q(i)$. Then note that $a + bi - a - bi = 0 = - a - bi + a
  + bi$, so if $x \in \Q(i)$, the additive inverse is simply $-x$. Thus $\Q(i)$
  is a subgroup of $\C$ under addition in $\C$.

  Second, we must show that $\Q(i) - \{0\}$ is a subgroup of $\C - \{0\}$
  under multiplication in $\C$. Consider the product
  \begin{align*}
    xy
    &= (a + bi)(c + di) \\
    &= ac + adi + bci - bd \\
    &= ac - bd + i(ad + bc)\, .
  \end{align*}
  Then $xy \in \Q(i) - \{0\}$, and thus $\Q(i) - \{0\}$ is closed under
  multiplication in $\C$. Now note that the multiplicative identity $1$ in $\C$
  is also in $\Q(i)$. Now consider the multiplicative inverse of $x$,
  \begin{align*}
    x^{-1}
    &= {1 \over x} \\
    &= {1 \over a + bi} \\
    &= {1 \over a + bi} \cdot {a - bi \over a - bi} \\
    &= {a - bi \over a^2 + b^2} \\
    &= {a \over a^2 + b^2} - {bi \over a^2 + b^2} \\
    &= {a \over a^2 + b^2} - {b \over a^2 + b^2} \cdot i\, .
  \end{align*}
  Then for any $x \in \Q(i) - \{0\}$, $x^{-1} \in \Q(i) - \{0\}$. Thus $\Q(i) -
  \{0\}$ is a subgroup of $\R - \{0\}$ under multiplication in $\R$.
\end{proof}

\subsection{Part b)}

\emph{Claim.} $\Q(i)$ is the smallest subfield of $\C$ containing $i$.

\begin{proof}
  Let $F$ be a subfield of $\C$ containing $i$. Then $1,i \in F$, and since $F$
  is an abelian subgroup under addition in $\C$, $F$ must contain $\Z(i)$. Since
  $F - \{0\}$ is an abelian subgroup under multiplication in $\C$, $F$ must
  contain multiplicative inverses for all elements of $\Z(i) - \{0\}$, so $F$
  contains all elements of the form $1/a : a \in \Z(i) - \{0\}$. Then since $F$
  is again an abelian subgroup under addition in $\C$, including elements of the
  form $1/a : a \in \Z(i) - \{0\}$ implies that $F$ contains $\Q(i)$. Since any
  subfield of $\Q(i)$ must contain $0, 1, i$ (and thus all $\Z(i)$), and their
  inverses (and thus all $\Q(i)$), we can conclude that $\Q(i)$ has no proper
  subfields. Thus $\Q(i)$ must be the smallest subfield of $\C$ containing $i$.
\end{proof}

\section{Exercise 17.22}

Let $\Q(\sqrt 2) = \{a + b\sqrt 2 : a,b \in \Q\}$.

\subsection{Part a)}

\emph{Claim.} Let $a,b \in \Q$. Then $a + b\sqrt 2 = 0$ if and only if $a = b =
0$.

\begin{proof}
  Assume $a = b =0$. Then
  \begin{align*}
    a + b \sqrt 2 &= 0 + 0 \cdot \sqrt 2 \\
                  &= 0\, .
  \end{align*}

  Assume $a + b \sqrt 2 = 0$, and for the purpose of contradiction assume $a
  \neq b \neq 0$. Then

  \begin{align*}
    &a + b\sqrt 2 = 0 \\ \Longleftrightarrow \quad
    &a = -b \sqrt{2}\, .
  \end{align*}
  Then $a \notin \Q$ unless $b = 0$ or $b = \pm \sqrt 2$, but we have that $b \in \Q$
  and $b \neq 0$, so we have found a contradiction. Thus $a = b = 0$.
\end{proof}

\subsection{Part b)}

\emph{Claim.} $\Q(\sqrt 2)$ is a subfield of $\R$.

\begin{proof}
  Let $x = a + b \sqrt 2$ and $y = c + d \sqrt 2$ where $a,b,c,d \in \Q$. Then
  \begin{align*}
    x + y
    &= a + b \sqrt 2 + c + d \sqrt 2 \\
    &= a + c + \sqrt 2 (b + d)\, .
  \end{align*}
  Thus $x + y \in \Q(\sqrt 2)$, so $\Q(\sqrt 2)$ is closed under addition in
  $\R$. Note that $0 \in \Q(\sqrt 2)$. Since $a + b \sqrt 2 - a -b \sqrt 2 = 0 =
  -a - b \sqrt 2 + a + b \sqrt 2$, for any $x \in \Q(\sqrt 2)$ the additive
  inverse is $-x \in \Q(\sqrt 2)$. Thus $\Q(\sqrt 2)$ is a subgroup of $\R$
  under addition in $\R$.

  Consider the product
  \begin{align*}
    xy
    &= (a + b \sqrt 2)(c + d \sqrt 2) \\
    &= ac + 2bd + \sqrt 2 (ad + bc)\, .
  \end{align*}
  Thus $xy \in \Q(\sqrt 2) - \{0\}$, so $\Q(\sqrt 2) - \{0\}$ is closed under
  multiplication in $\R$. Note that $1 \in \Q(\sqrt 2)$. Now consider the
  multiplicative inverse of $x$,
  \begin{align*}
    x^{-1}
    &= {1 \over x} \\
    &= {1 \over a + b \sqrt 2} \\
    &= {1 \over a + b \sqrt 2} \cdot {a - b \sqrt 2 \over a - b \sqrt 2} \\
    &= {a - b \sqrt 2 \over a^2 - 2b^2} \\
    &= {a \over a^2 - 2b^2} - {b \sqrt 2 \over a^2 - 2b^2} \\
    &= {a \over a^2 - 2b^2} - {b \over a^2 - 2b^2} \cdot \sqrt 2\, .
  \end{align*}
  Then for any $x \in \Q(\sqrt 2) - \{0\}$, $x^{-1} \in \Q(\sqrt 2) - \{0\}$.
  Thus $\Q(\sqrt 2) - \{0\}$ is a subgroup of $\R - \{0\}$ under multiplication
  in $\R$.
\end{proof}

\subsection{Part c)}

\emph{Claim.} $\Q(\sqrt 2)$ is a proper subfield of $\R$.

\begin{proof}
  Since we have already proved that $\Q(\sqrt 2)$ is a subfield of $\R$, it
  suffices to show that $\Q(\sqrt 2)$ is a strict subset of $\R$.

  Assume for the purpose of contradiction that $\sqrt{3} \in \Q(\sqrt 2)$. Then
  there exists $a,b \in \Q$ where
  \begin{align*}
    &a + b \sqrt 2 = \sqrt 3 \\ \Longleftrightarrow \quad
    &\sqrt 3(a + b \sqrt 2) = \sqrt 3 \cdot \sqrt 3 \\ \Longleftrightarrow \quad
    &a \sqrt 3 + b \sqrt 6 = 3 \\ \Longleftrightarrow \quad
    &a (a + b \sqrt 2) + b \sqrt 2 = 3 \\ \Longleftrightarrow \quad
    &a^2 + ab\sqrt 2 + b \sqrt 2 = 3 \\ \Longleftrightarrow \quad
    & \sqrt 2 (ab + b) = 3 - a^2\, .
  \end{align*}
  Then on the right-hand side we have a rational expression, but on the
  left-hand side we can only have a rational when $ab + b = 0$. Then $3 - a^2 =
  0 \Rightarrow a^2 = 3 \Rightarrow a = \pm \sqrt 3$. But we have that $a \in
  \Q$, thus we have found a contradiction.
\end{proof}

\subsection{Part d)}

\emph{Claim.} $\Q(\sqrt 2)$ is a field lying strictly between $\Q$ and $\R$.

\begin{proof}
  We have already proved that $\Q(\sqrt 2)$ is a field, and we have proved the
  $\Q(\sqrt 2)$ is a strict subset of $\R$. Clearly $\Q$ is a strict subset of
  $\Q(\sqrt 2)$. Then $\Q(\sqrt 2)$ is a field lying strictly between $\Q$ and
  $\R$.
\end{proof}

\subsection{Part e)}

\emph{Claim.} $\Q(\sqrt 2)$ is the smallest subfield of $\R$ containing $\sqrt
2$.

\begin{proof}
  Let $F$ be a subfield of $\R$ containing $\sqrt 2$. Then $1, \sqrt 2 \in F$,
  and since $F$ is an abelian subgroup under addition in $\R$, $F$ must contain
  $\Z(\sqrt 2)$. Since $F - \{0\}$ is an abelian subgroup under multiplication
  in $\C - \{0\}$, $F$ must contain multiplicative inverses for all elements of
  $\Z(\sqrt 2) - \{0\}$, so $F$ contains all elements of the form $1/a : a \in \Z(\sqrt
  2) - \{0\}$. Then since $F$ is again an abelian subgroup under addition in
  $\R$, including elements of the form $1/a : a \in \Z(\sqrt 2)$ implies that
  $F$ contains $\Q(\sqrt 2)$. Since any subfield of $\Q(\sqrt 2)$ must contain
  $0, 1, \sqrt 2$ (and thus all $\Z(\sqrt 2)$), and their inverses (and thus all
  $\Q(\sqrt 2)$), we can conclude that $\Q(\sqrt 2)$ has no proper subfields.
  Thus $\Q(\sqrt 2)$ must be the smallest subfield $\R$ containing $\sqrt 2$.
\end{proof}

\section{Exercise 23}

Let $F$ be any subfield of $\C$ containing $\R$ as a proper subfield.

\subsection{Part a)}

\emph{Claim.} $i \in F$.

\begin{proof}
  Since $\R < F \leq \C$, there exists $x \in F$ such that $x \notin \R$ and $x
  \in \C$. Then $x = a + bi$ for $a,b \in \R$, and since $x \notin \R$, $b \neq
  0$. Since $F$ is a field, $x - a \in F$, and similarly $(x - a) / b \in F$.
  Then since $i = (x - a) / b$, $i \in F$.
\end{proof}

\subsection{Part b)}

\emph{Claim.} $F = \C$.

\begin{proof}
  We have that $F$ contains $\R \cup \{i\}$. Since $F$ is an abelian subgroup
  under addition in $\C$, $F$ contains all elements of the form $nx : x \in F, n
  \in \Z$. Furthermore, since $F$ is an abelian subgroup under multiplication in
  $\C$, $F$ also contains all elements of the form $y^m : y \in F, m \in \Z$.
  Since $F$ is closed under addition in $\C$, we have that $F$ contains all
  elements of the form $nx + y^m : x,y \in F, n,m \in \Z$. Thus $F = \C$.
\end{proof}

\subsection{Part c)}

\emph{Claim.} There are no fields lying strictly between the fields $\R$ and
$\C$.

\begin{proof}
  Since any subfield of $\C$ containing $\R$ as a proper subfield must equal
  $\C$, there are no fields lying strictly between the fields $\R$ and $\C$.
\end{proof}

\section{Exercise 19.5}

\subsection{Part a)}

\[
\polylongdiv{X^5 - 2X^3 + 2X^2 - 3}{X - 1}\, .
\]

Thus the quotient is $X^4 + X^3 - X^2 + 1$ and the remainder is $-2$.

\subsection{Part b)}

\[
\polylongdiv{X^5 - 2X^3 + 2X^2 - 3}{2X^3 - 1}\, .
\]

Thus the quotient is ${1 \over 2} X^2 - 1$ and the remainder is ${5 \over 2} X^2
- 4$.

\subsection{Part c)}

The quotient is $0$ and the remainder is $2X^3 - 1$.

\subsection{Part d)}

The quotient is $X^2 + 1$ and the remainder is $X^2 + 1$.

\subsection{Part e)}

The quotient is $0$ and the remainder is $X^5 - 2X^3 + 2X - 3$.

\section{Exercise 19.7}

Let $R$ be a ring.

\medskip
\noindent
\emph{Claim.} Multiplication in $R[X]$ distributes over addition in $R[X]$.

\begin{proof}
  Without left-distributing we have
  \begin{align*}
    \left( \sum_{n=0}^{< \infty} a_n X^n \right) \left( \sum_{n=0}^{< \infty} b_n X^n +
    \sum_{n=0}^{< \infty} c_n X^n \right)
    &= \left( \sum_{n=0}^{< \infty} a_n X^n \right) \sum_{n=0}^{< \infty}
      (b_n + c_n) X^n \\
    &= \sum_{n=0}^{< \infty} \left( \sum_{i+j=n} a_i(b_j +c_j) \right) X^n \, .
  \end{align*}
  With left-distributing we have
  \begin{align*}
    \left( \sum_{n=0}^{< \infty} a_n X^n \right) \left( \sum_{n=0}^{< \infty} b_n X^n +
    \sum_{n=0}^{< \infty} c_n X^n \right)
    &= \left( \sum_{n=0}^{< \infty} a_n X^n \right)
      \left( \sum_{n=0}^{< \infty} b_n X^n \right) +
      \left( \sum_{n=0}^{< \infty} a_n X^n \right)
      \left( \sum_{n=0}^{< \infty} c_n X^n \right) \\
    &= \sum_{n=0}^{< \infty} \left( \sum_{i+j=n} a_i b_j \right) X^n +
      \sum_{n=0}^{< \infty} \left( \sum_{i+j=n} a_i c_j \right) X^n \\
    &= \sum_{n=0}^{< \infty} \left( \sum_{i+j=n} a_i b_j + a_i c_j \right) X^n \\
    &= \sum_{n=0}^{< \infty} \left( \sum_{i+j=n} a_i(b_j + c_j) \right) X^n\, .
  \end{align*}

  Without right-distributing we have
  \begin{align*}
    \left( \sum_{n=0}^{< \infty} b_n X^n + \sum_{n=0}^{< \infty} c_n X^n \right)
    \sum_{n=0}^{< \infty} a_n X^n
    &= \left( \sum_{n=0}^{< \infty} (b_n + c_n) X^n \right) \sum_{n=0}^{< \infty} a_n X^n \\
    &= \sum_{n=0}^{< \infty} \left( \sum_{i+j=n} (b_i +c_i) a_j \right) X^n \, .
  \end{align*}
  With right-distributing we have
  \begin{align*}
    \left( \sum_{n=0}^{< \infty} b_n X^n + \sum_{n=0}^{< \infty} c_n X^n \right)
    \sum_{n=0}^{< \infty} a_n X^n
    &= \left( \sum_{n=0}^{< \infty} b_n X^n \right)
      \left( \sum_{n=0}^{< \infty} a_n X^n \right) +
      \left( \sum_{n=0}^{< \infty} c_n X^n \right)
      \left( \sum_{n=0}^{< \infty} a_n X^n \right) \\
    &= \sum_{n=0}^{< \infty} \left( \sum_{i+j=n}  b_i a_j \right) X^n +
      \sum_{n=0}^{< \infty} \left( \sum_{i+j=n} c_i a_j \right) X^n \\
    &= \sum_{n=0}^{< \infty} \left( \sum_{i+j=n} b_i a_j + c_i a_j \right) X^n \\
    &= \sum_{n=0}^{< \infty} \left( \sum_{i+j=n} (b_i + c_i) a_j \right) X^n\, .
  \end{align*}

  Thus multiplication in $R[X]$ distributes over addition in $R[X]$.
\end{proof}

\noindent
\emph{Claim.} Multiplication in $M_n(R)$ distributes over addition in $M_n(R)$.

\begin{proof}
  Without left-distributing we have
  \begin{align*}
    (a_{ij})((b_{ij}) + (c_{ij}))
    &= (a_{ij})(b_{ij} + c_{ij}) \\
    &= (d_{ij})
  \end{align*}
  where $d_{ij} = \sum_{k=1}^{n} a_{ik}(b_{kj} + c_{kj})$ for all $i$ and $j$.
  With left-distributing we have
  \begin{align*}
    (a_{ij})((b_{ij}) + (c_{ij}))
    &= (a_{ij})(b_{ij}) + (a_{ij})(c_{ij}) \\
    &= (d_{ij}) + (e_{ij}) \\
    &= (d_{ij} + e_{ij})
  \end{align*}
  where $d_{ij} = \sum_{k=1}^n a_{ik} b_{kj}$, and $e_{ij} = \sum_{k=1}^n a_{ik}
  c_{kj}$ for all $i$ and $j$, so
  \begin{align*}
    (d_{ij} + e_{ij})
    &= \sum_{k=1}^n a_{ik} b_{kj} + \sum_{k=1}^n a_{ik} c_{kj} \\
    &= \sum_{k=1}^n a_{ik} b_{kj} + a_{ik} c_{kj} \\
    &= \sum_{k=1}^n a_{ik} (b_{kj} + c_{kj})
  \end{align*}
  for all $i$ and $j$.

  Without right-distributing we have
  \begin{align*}
    ((b_{ij}) + (c_{ij}))(a_{ij})
    &= (b_{ij} + c_{ij})(a_{ij}) \\
    &= (d_{ij})
  \end{align*}
  where $d_{ij} = \sum_{k=1}^n (b_{ik} + c_{ik}) a_{kj}$ for all $i$ and $j$.
  With right-distributing we have
  \begin{align*}
    ((b_{ij}) + (c_{ij}))(a_{ij})
    &= (b_{ij})(a_{ij}) + (c_{ij})(a_{ij}) \\
    &= (d_{ij}) + (e_{ij}) \\
    &= (d_{ij} + e_{ij})
  \end{align*}
  where $d_{ij} = \sum_{k=0}^n b_{ik} a_{kj}$, and $e_{ij} = \sum_{k=1}^n c_{ik}
  a_{kj}$ for all $i$ and $j$, so
  \begin{align*}
    (d_{ij} + e_{ij})
    &= \sum_{k=1}^n b_{ik} a_{kj} + \sum_{k=1}^n c_{ik} a_{kj} \\
    &= \sum_{k=1}^n b_{ik} a_{kj} + c_{ik} a_{kj} \\
    &= \sum_{k=1}^n (b_{ik} + c_{ik}) a_{kj}
  \end{align*}
  for all $i$ and $j$.

  Thus multiplication in $M_n(R)$ distributes over addition in $M_n(R)$.
\end{proof}

\section{Exercise 19.12}

\subsection{Part a)}

\emph{Claim.} $1 + 2X$ is a unit in $\Z_4[X]$.

\begin{proof}
  Note that $1 + 2X + 3 + 2X = 0 = 3 + 2X + 1 + 2X$, so $3 + 2X$ is the additive
  inverse of $1 + 2X$. Now note that $(1 + 2X)(1 + 2X) = 1$, so $1 + 2X$ is its
  own multiplicative inverse. Since we have found both inverses in $\Z_4[X]$, we conclude
  that $1 + 2X$ is a unit in $\Z_4[X]$.
\end{proof}

\noindent
\emph{Claim.} $1 + 3X$ is a unit in $\Z_9[X]$.

\begin{proof}
  Note that $1 + 3X + 8 + 6X = 0 = 8 + 6X + 1 + 3X$, so $8 + 6X$ is the additive
  inverse of $1 + 3X$. Now note that $(1 + 3X)(1 + 3X) = 1$, so $1 + 3X$ is its
  own multiplicative inverse. Since we have found both inverses in $\Z_9[X]$, we
  conclude that $1 + 3X$ is a unit in $\Z_9[X]$.
\end{proof}

\subsection{Part b)}

\emph{Claim.} $1 + 3X$ is not a unit in $\Z_4[X]$.

\begin{proof}
  Assume for the purpose of contradiction that $1 + 3X$ is a unit in $\Z_4[X]$.
  Then
  \begin{align*}
    (1 + 3X) \sum_{n=0}^{1} a_n X^n
    &= 1 \\
    &= \sum_{n=0}^{1} \left( \sum_{i+j=n} b_i a_j \right) X^n
  \end{align*}
  for some $\sum_{n=0}^{1} a_n X^n \in \Z_4[X]$ and $b_0 = 1$, $b_1 = 3$. (Note
  that we can terminate our sum at $n = 1$ since the $1$ in $1 + 3X$ guarantees
  that any $a_n X^n$ for $n > 1$ and $a_n \neq 0$ terms in the sum would then
  appear on the right-hand side of our equation.) Then $b_0 a_0 = a_0 = 1$ and
  $b_0 a_1 + b_1 a_0 = a_1 + 3 = 0$. Then $a_1 = 1$. But it is easy to verify
  that $(1 + 3X)(1 + X) = 1 + 3X^2 \neq 1$, thus we have found a contradiction.
\end{proof}

\noindent
\emph{Claim.} $1 + 2X$ is not a unit in $\Z_9[X]$.

\begin{proof}
  Assume for the purpose of contradiction that $1 + 2X$ is a unit in $\Z_9[X]$.
  Then
  \begin{align*}
    (1 + 2X) \sum_{n=0}^{1} a_n X^n
    &= 1 \\
    &= \sum_{n=0}^{1} \left( \sum_{i+j=n} b_i a_j \right) X^n
  \end{align*}
  for some $\sum_{n=0}^1 a_n X^n \in \Z_9[X]$ and $b_0 = 1, b_1 = 2$. (Note that
  we can terminate our sum at $n = 1$ since the $1$ in $1 + 2X$ guarantees that
  any $a_n X^n$ for $n > 1$ and $a_n \neq 0$ terms in the sum would then appear
  on the right-hand side of our equation.) Then $b_0 a_0 = a_0 = 1$ and $b_0 a_1
  + b_1 a_0 = a_1 + 2 = 0$. Then $a_1 = 7$. But it is easy to verify that $(1 +
  2X) (1 + 7X) = 1 + 5X \neq 1$, thus we have found a contradiction.
\end{proof}

\end{document}

%  LocalWords:  subring quaternions subfield subfields abelian
