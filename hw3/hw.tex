\documentclass{abrice}

\title{Math 393: Homework 3}
\author{Anthony Brice}

\usepackage[mathscr]{eucal}

\newcommand{\GL}{\text{GL}}
\newcommand{\Z}{\mathbb{Z}}
\newcommand{\R}{\mathbb{R}}
\newcommand{\Q}{\mathbb{Q}}
\renewcommand{\P}{\mathscr{P}}

\usepackage{amsthm}

\begin{document}
\maketitle

\section{Exercise 18.2}

\emph{Claim.} Let $a$ and $b$ be elements of commutative ring $R$. Then ${(a + b)}^2 = a^2
+ 2 \cdot (ab) + b^2$.

\begin{proof}
  \begin{align*}
    {(a + b)}^2
    &= (a + b) \cdot (a + b) \\
    &= a^2 + ab + ba + b^2 \\
    &= a^2 + 2 \cdot (ab) + b^2\, . \qedhere
  \end{align*}
\end{proof}

\noindent
\emph{Claim.} Let $a$ and $b$ be elements of non-commutative ring $R$. Then it
does not necessarily hold that ${(a + b)}^2 = a^2 + 2 \cdot (ab) + b^2$.

\begin{proof}
  Let $a = \left(\begin{smallmatrix} 1&1 \\ 0&1 \end{smallmatrix} \right)$ and
  $b = \left( \begin{smallmatrix} 0&1 \\ 0&1 \end{smallmatrix} \right)$, both in
  $M_2(\Q)$. Then
  \begin{align*}
    {(a + b)}^2
    &= (a + b) \cdot (a + b) \\
    &= a^2 + ab + ba + b^2 \\
    &= \left(  \begin{smallmatrix} 1&2 \\ 0&1 \end{smallmatrix} \right)
      + \left( \begin{smallmatrix} 0&2 \\ 0&1 \end{smallmatrix} \right)
      + \left( \begin{smallmatrix} 0&1 \\ 0&1 \end{smallmatrix} \right)                                       + \left( \begin{smallmatrix} 0&1 \\ 0&1 \end{smallmatrix} \right) \\
    &\neq a^2 + 2 \cdot (ab) + b^2\, . \qedhere
  \end{align*}
\end{proof}

\section{Exercise 18.6}

\emph{Claim.} Every subring of an integral domain is an integral domain.

\begin{proof}
  Let $S$ be a subring of ring $R$ where $R$ is an integral domain. Then $R$ is
  a ring that is a subring of some field $T$. Then $S$ is also a subring of $T$,
  thus satisfying the definition of an integral domain.
\end{proof}

\section{Exercise 18.13}

\emph{Claim.} For any $x$ and $y$ in integral domain $R$, $x^2 = y^2$ if and
only if $x = y$ or $x = -y$.

\begin{proof}
  Let $x,y \in R$ where $R$ is an integral domain.

  Assume $x = y$. Then $x^2 = xx = yy = y^2$. Similarly for $x = -y$, $x^2 = xx
  = (-y)(-y) = y^2$ by Proposition 18.18.

  Assume $x^2 = y^2$, and assume that $x \neq y$ and $x \neq -y$ for the purpose
  of contradiction.
  \begin{align*}
    x^2 &= y^2 \\
    \Longleftrightarrow \quad xx &= yy\, .
  \end{align*}
  \emph{TODO:} ???
\end{proof}

\section{Exercise 18.15}

Let $R = \{a/b : a,b \in \Z \text{ and } b \text{ is odd}\}$.

\subsection{Part a)}

\emph{Claim.} $R$ is a subring of $\Q$.

\begin{proof}
  We want to show that the three conditions of Proposition 18.13 hold.

  Clearly $R$ is a subset of $\Q$. Let $x,y \in R$ and consider $z = x + y$.
  Since the least common multiple of two odd numbers is itself an odd number, $z
  \in R$ as well, so $R$ is closed under the group operation of $\Q$. Since $0
  \in \Z$ and $0/b$ where $b$ is odd equals $0$, we have the additive identity
  element in $R$. By definition, $R$ clearly gives us additive inverses. Thus
  $R$ is a subgroup of $\Q$.

  Since $1 \in \Z$ and is odd, we have the multiplicative identity in $R$.

  Now consider the product $w = xy$. Since the product of two odd numbers is odd
  and the product of any two integers is an integer, $w \in R$ and thus $R$ is
  closed under multiplication in $\Q$.
\end{proof}

\subsection{Part b)}

\emph{Claim.} $R^* = \{ a/b : a,b \in \Z \text{ and } a,b \text{ are odd} \}$.

\begin{proof}
  Let $a/b \in R$. Then the multiplicative inverse of $a/b$ is $b/a$. Then $b/a$
  is only in $R$ when $a$ is odd.
\end{proof}

\section{Exercise 18.18}

Let $\Z[\sqrt 2] = \{a + b\sqrt 2 : a,b \in \Z \}$.

\subsection{Part a)}

\emph{Claim.} $\Z[\sqrt 2]$ is a subring of $\R$.

\begin{proof}
  Let $x = a + b\sqrt 2$ and $y = c + d\sqrt{2}$ where $a,b,c,d \in \Z$. Then
  \begin{align*}
    x + y
    &= a + b\sqrt 2 + c + d\sqrt 2 \\
    &= a + c + (b + d)\sqrt 2\, .
  \end{align*}
  Thus $x + y \in \Z[\sqrt 2]$. Clearly $0 \in \Z[\sqrt 2]$, and we have
  additive inverses since for any $a,b \in \Z$ we have $-a, -b \in \Z$.

  Consider the element $1 + 0 \sqrt 2 = 1$. Then $1 \in \Z[\sqrt 2]$.

  Next consider the product
  \begin{align*}
    xy
    &= (a + b\sqrt 2) (c + d \sqrt 2) \\
    &= a c + a d \sqrt 2 + b c \sqrt 2 + (b \sqrt 2)(d \sqrt 2) \\
    &= a c + \sqrt 2 (ad + bc) + 2bd \\
    & = ac + 2bd + \sqrt 2(ad + bc)\, .
  \end{align*}
  Thus $xy \in \Z[\sqrt 2]$.
\end{proof}

\subsection{Part b)}

\emph{Claim.} $\Z[\sqrt 2]$ is the smallest subring of $\R$ containing $\sqrt
2$.

\begin{proof}
  \emph{TODO:} Show that any subring containing $\sqrt 2$ contains all elements
  in $\Z[\sqrt 2]$.
\end{proof}

\subsection{Part c)}

\emph{Claim.} $1 + \sqrt 2$ is a unit in $\Z[\sqrt 2]$.

\begin{proof}
  \begin{align*}
    (1 + \sqrt 2)(\sqrt 2 - 1)
    &= \sqrt 2 - 1 + 2 - \sqrt 2 \\
    &= 1\, .
  \end{align*}
  Since $\sqrt 2 - 1 \in \Z[\sqrt 2]$, the element is the inverse of $1 + \sqrt 2$.
\end{proof}

\section{Exercise 18.20}

Let $\Z[\sqrt[3]{2}] = \{ a + b \sqrt[3]{2} + c \sqrt[3]{4} : a,b,c \in \Z \}$.

\subsection{Part a)}

\emph{Claim.} $\Z[\sqrt[3]{2}]$ is a subring of $\R$.

\begin{proof}
  Let $x = a + b \sqrt[3]{2} + c \sqrt[3]{4}$ and $y = e + f \sqrt[3]{2} + g
  \sqrt[3]{4}$ where $a,b,c,d,e,f \in \Z$. Then
  \begin{align*}
    x + y
    &= a + b \sqrt[3]{2} + c \sqrt[3]{4} + e + f \sqrt[3]{2} + g \sqrt[3]{4} \\
    &= a + e + \sqrt[3]{2}(b + f) + \sqrt[3]{4}(c + g)\, .
  \end{align*}
  So $x + y \in \Z[\sqrt[3]{2}]$. Clearly $0 \in \Z[\sqrt[3]{2}]$, and we have
  additive inverses since for any $a,b,c \in \Z$ we have $-a,-b,-c \in \Z$.

  Consider $1 + 0\sqrt[3]{2} + 0\sqrt[3]{4} = 1$. Then $1 \in \Z[\sqrt[3]{2}]$.

  Now consider the product
  \begin{align*}
    xy
    &= (a + b \sqrt[3]{2} + c \sqrt[3]{4})(e + f \sqrt[3]{2} + g \sqrt[3]{4}) \\
    &= ae + af \sqrt[3]{2} + ag \sqrt[3]{4} + be \sqrt[3]{2} + bf \sqrt[3]{4}
      + 2bg + ce \sqrt[3]{4} + 2cf + 2cg \sqrt[3]{2} \\
    &= ae + 2(bg + cf) + \sqrt[3]{2} (af + be + 2cg) + \sqrt[3]{4} (ag + bf + ce)\, .
  \end{align*}
  So $xy \in \Z[\sqrt[3]{2}]$.
\end{proof}

\subsection{Part b)}

\emph{Claim.} Let $S = \{ a + b \sqrt[3]{2} : a,b \in \Z \}$. Then $\sqrt[3]{4}
\notin S$.

\begin{proof}
  Assume $\sqrt[3]{4} \in S$. Then there exists $a,b \in \Z$ where
  \begin{align*}
    &a + b \sqrt[3]{2} = \sqrt[3]{4} \\
    \Longleftrightarrow \quad
    &a = \sqrt[3]{4} - b \sqrt[3]{2} \\
    \Longleftrightarrow \quad
    &a = \sqrt[3]{2} (\sqrt[3]{2} - b)\, .
  \end{align*}
  \emph{TODO:} Show $a$ or $b$ is not in $\Z$.
\end{proof}

\subsection{Part c)}

\emph{Claim.} Let $S = \{a + b\sqrt[3]{2} : a,b \in \Z \}$. Then $S$ is not a
subring of $\R$.

\begin{proof}
  Consider that $\sqrt[3]{2} \cdot \sqrt[3]{2} = \sqrt[3]{4}$ is not in $S$.
  Then $S$ is not closed under multiplication in $\R$.
\end{proof}

\subsection{Part d)}

\emph{Claim.} $\Z[\sqrt[3]{2}]$ is the smallest subring of $\R$ containing
$\sqrt[3]{2}$.

\begin{proof}
  \emph{TODO:} Show that any subring containing $\sqrt[3]{2}$ contains all
  elements in $\Z[\sqrt[3]{2}]$.
\end{proof}

\section{Exercise 18.28}

Let $a$ and $b$ be elements of a ring $R$ such that $ab = ba$.

\subsection{Part a)}

\emph{Claim.} ${(a + b)}^2 = a^2 + 2 \cdot (ab) + b^2$.

\begin{proof}
  \begin{align*}
    {(a + b)}^2
    &= (a + b) \cdot (a + b) \\
    &= a^2 + ab + ba + b^2 \\
    &= a^2 + 2 \cdot (ab) + b^2\, . \qedhere
  \end{align*}
\end{proof}

\subsection{Part b)}

\emph{Claim.} ${(a + b)}^3 = a^3 + 3 \cdot (a^2b) + 3 \cdot (ab^2) + b^3$.

\begin{proof}
  \begin{align*}
    {(a + b)}^3
    &= (a + b){(a + b)}^2 \\
    &= (a + b)(a^2 + 2ab + b^2) \\
    &= a^3 + 2a^2b + ab^2 + a^2b + 2ab^2 + b^3 \\
    &= a^3 + 3 \cdot (a^2b) + 3 \cdot (ab^2) + b^3\, . \qedhere
  \end{align*}
\end{proof}

\subsection{Part c)}

\emph{Claim.} The binomial theorem
\[
  {(a + b)}^n = \sum_{k=0}^n {n \choose k} \cdot (a^k b^{n-k})
\]
holds for any positive integer $n$.

\begin{proof}
  Consider the base case $n = 1$. We have on the left-hand side $a + b$, and
  computing the right gives us
  \begin{align*}
    \sum_{k=0}^1 {1 \choose k} \cdot (a^k b^{1-k})
    &= {1 \choose 0} \cdot (a^0b^1) + {1 \choose 1} \cdot (a^1b^0) \\
    &= b + a \\
    &= a + b\, .
  \end{align*}

  Now assume for the purpose of induction that the claim holds when $n = m$ and
  consider the case when $n = m + 1$. Starting from the right-hand side we have
  \begin{align*}
    \sum_{k=0}^{m+1} {m + 1 \choose k} \cdot (a^k b^{m+1-k})
    &= \sum_{k=0}^{m+1} \left(  {m \choose k} + {m \choose k-1 } \right) \cdot
      (a^k b^{m+1-k}) \\
    &= \sum_{k=0}^{m+1} {m \choose k} \cdot (a^k b^{m+1-k}) + {m \choose k-1}
      \cdot (a^k b^{m+1-k}) \\
    &= \sum_{k=0}^{m+1} {m \choose k} \cdot (a^k b^{m+1-k}) + \sum_{j=0}^{m+1} {m \choose j-1}
      \cdot (a^j b^{m+1-j})\, .
  \end{align*}
  Since ${m \choose m + 1} = 0$ and ${m \choose -1} = 0$,
  \begin{align*}
    \sum_{k=0}^{m+1} {m + 1 \choose k} \cdot (a^k b^{m+1-k})
    &= \sum_{k=0}^{m} {m \choose k} \cdot (a^k b^{m+1-k}) + \sum_{j=1}^{m+1} {m \choose j-1}
      \cdot (a^j b^{m+1-j}) \\
    &= b \cdot \sum_{k=0}^{m} {m \choose k} \cdot (a^k b^{m-k}) +
      \sum_{j=1}^{m+1} {m \choose j-1}
      \cdot (a^j b^{m+1-j}) \\
    &= b \cdot {(a + b)}^m + \sum_{j=1}^{m+1} {m \choose j-1} \cdot (a^j
      b^{m+1-j}) \\
    &= b \cdot {(a + b)}^m + a \cdot \sum_{j=1}^{m+1} {m \choose j-1} \cdot
      (a^{j-1} b^{m-(j-1)}) \\
    &= b \cdot {(a + b)}^m + a \cdot \sum_{j=0}^{m} {m \choose j} \cdot
      (a^{j} b^{m-j}) \\
    &= b \cdot {(a + b)}^m + a \cdot {(a + b)}^m \\
    &= (b + a){(a + b)}^m \\
    &= {(a + b)}^{m+1}\, . \qedhere
  \end{align*}
\end{proof}

\section{Exercise 18.29}

Let $S$ be a set, and let $\P(S)$ be the set of all subsets of $S$. Define $A
\oplus B = A \cup B - A \cap B$ or any two sets $A$ and $B$.

\end{document}

%  LocalWords:  subring
